\documentclass[aps,rmp,twocolumn,amsmath,amssymb,nofootinbib,superscriptaddress]{revtex4}

\newcommand{\bra}[1]{\langle#1|}
\newcommand{\ket}[1]{|#1\rangle}
\newcommand{\op}[2]{\hat{\textbf{#1}}_{#2}}
\newcommand{\dagop}[2]{\hat{\textbf{#1}}_{#2}^\dag}
\usepackage[pdftex]{graphicx}
\usepackage{mathrsfs}
\usepackage[colorlinks]{hyperref}
\usepackage[dvipsnames]{xcolor}

\newcommand{\sihui}[1]{{\color{Orchid}{#1}}}
\newcommand{\peter}[1]{{\color{YellowGreen}{#1}}}
\newcommand{\comment}[1]{{\color{blue}{#1}}}

\begin{document}

\bibliographystyle{apsrev}

%
% Title
%

\title{Linear Optics Interferometry}

%
% Authors
%

\author{Si-Hui Tan}
\email[]{sihui\_tan@sutd.edu.sg}
\affiliation{Singapore University of Technology and Design, 8 Somapah Road, Singapore}

\author{Peter P. Rohde}
\email[]{dr.rohde@gmail.com}
\homepage{http://www.peterrohde.org}
\affiliation{Centre for Quantum Computation and Intelligent Systems (QCIS), Faculty of Engineering \& Information Technology, University of Technology Sydney, NSW 2007, Australia}

\date{\today}

\frenchspacing

%
% Abstract
%

\begin{abstract}
\end{abstract}

\maketitle

\tableofcontents

\section{Introduction}

\sihui{Si-Hui can colour code things she adds like this}

\peter{And Peter can do it like this}

\comment{Let's add comments and questions like this}

\section{Linear Optics Quantum Computation}

\section{Boson-sampling}

\section{Conclusion}

%
% Acknowledgments
%

\begin{acknowledgments}
\end{acknowledgments}

%
% Bibliography
%

\bibliography{paper}

\end{document}
