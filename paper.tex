\documentclass[aps,rmp,twocolumn,amsmath,amssymb,nofootinbib,superscriptaddress]{revtex4}

\newcommand{\bra}[1]{\langle#1|}
\newcommand{\ket}[1]{|#1\rangle}
\newcommand{\op}[2]{\hat{\textbf{#1}}_{#2}}
\newcommand{\dagop}[2]{\hat{\textbf{#1}}_{#2}^\dag}
\usepackage[pdftex]{graphicx}
\usepackage{mathrsfs}
\usepackage[colorlinks]{hyperref}
\usepackage[dvipsnames]{xcolor}

\newcommand{\sihui}[1]{{\color{Orchid}{#1}}}
\newcommand{\peter}[1]{{\color{YellowGreen}{#1}}}
\newcommand{\comment}[1]{{\color{blue}{#1}}}
\newcommand{\ah}{\hat{a}}
\newcommand{\adagh}{\hat{a}^\dag}

\begin{document}

\bibliographystyle{apsrev}

%
% Title
%

\title{The Resurgence of the Linear Optics Interferometer --- Recent Advances \& Applications}

%
% Authors
%

\author{Si-Hui Tan}
\email[]{sihui\_tan@sutd.edu.sg}
\affiliation{Singapore University of Technology and Design, 8 Somapah Road, Singapore}

\author{Peter P. Rohde}
\email[]{dr.rohde@gmail.com}
\homepage{http://www.peterrohde.org}
\affiliation{Centre for Quantum Software \& Information (CQSI), Faculty of Engineering \& Information Technology, University of Technology Sydney, NSW 2007, Australia}

\date{\today}

\frenchspacing

%
% Abstract
%

\begin{abstract}
\end{abstract}

\maketitle

\tableofcontents

\section{Introduction}

\sihui{Si-Hui can colour code things she adds like this}

\peter{And Peter can do it like this}

\comment{Let's add comments and questions like this}

\section{Mathematical background}
\comment{Mathematical representation for LO networks, and very basic background on quantum optics}\\
\sihui{A idealized single photon in a quantum interferometer is described by its creation operator $\adagh_{j}$, where $j$ is the label of the mode the photon is in  within the interferometer. The creation and annihilation operators satisfy the bosonic comutator relationship $[\ah_{j},\adagh_{k}]=\delta_{j,k}$. A similar commutator relationship can be  written up when more degrees of freedom, such as polarization, orbital angular momentum, and time-bins \cite{bib:Tillmann2015,bib:Bozinovic2013, bib:Nicolas2014, bib:Humphreys2013, bib:Donohue2013}, are present. When multiple photons are present, they experience quantum interference when all quantum labels are the same.

The action of a $2d$-port linear optical interferometer (with an equal number of input and output ports) is expressed as an application of unitary operations on the creation operators,

\begin{align}
b_i^{\dag}=\sum_{j=1}^d U_{ij}a_j^\dag \ ,
\end{align}
where $a_j^\dag$ and $b_i^\dag$ are the creation operators of a single input and output photon in the $j$-th and $i$-th modes respectively, and $U\in SU(d)$. All such transformation can be expressed as sequence of beamsplitters and waveplates \cite{bib:Reck1994}. In the case when photons have additional labels, for instance, if they have internal labels on top of spatial labels, it is also possible to derive an analogous decomposition, known as a cosine-sine decomposition \cite{bib:Dhand2015}, that realizes the unitary transformation on the photons into a sequence of beamsplitters and internal transformations. Toolkits using group theory are being developed to deal with partial distinguishabilities among interfering photons \cite{bib:Tan2013,bib:deGuise2014,bib:deGuise2015}. Others use
quantum-to-classical transitions to explain multiparticle interference \cite{bib:Ra2013}.

}



\section{Optical encoding of quantum information}

\subsection{Single-photons}

\subsubsection{Polarisation}

\subsubsection{Dual-rail}

\subsubsection{Time-bins}

\subsection{Continuous-variables}

\subsubsection{Coherent states}

\subsubsection{Squeezed states}

\section{Efficient circuit decompositions of linear optics networks}

\comment{Discuss the Reck et al. decomposition}

\section{Experimental implementation}

\subsection{State preparation}

\subsubsection{Single-photons}

\subsubsection{Bell pairs}

\subsubsection{Coherent states}

\subsubsection{Squeezed states}

\subsection{Linear optics networks}

\subsubsection{Bulk-optics}

\subsubsection{Waveguides}

\subsubsection{Time-bins}

\comment{Discuss fibre-loop architecture}

\subsection{Measurement}

\subsubsection{Photodetection}

\comment{Discuss number-resolved and bucket detectors, multiplexed detection, APDs, current micropillar detectors}

\subsubsection{Homodyning}

\section{Applications for linear optics interferometry}

\subsection{Linear optics quantum computation}

\subsection{Boson-sampling}

\subsection{Quantum metrology}

\comment{Discuss NOON states - Heisenberg limited}

\comment{Discuss MORDOR scheme}

\subsection{Encrypted quantum computation}

\section{State of the art}

\comment{Discuss where experiments are at at the moment}

\section{Conclusion}

%
% Acknowledgments
%

\begin{acknowledgments}
P.P.R. is funded by an ARC Future Fellowship (project FT160100397).
\end{acknowledgments}

%
% Bibliography
%

\bibliography{paper}

\end{document}
