\documentclass[aps,rmp,twocolumn,amsmath,amssymb,nofootinbib,superscriptaddress]{revtex4}

\newcommand{\bra}[1]{\langle#1|}
\newcommand{\ket}[1]{|#1\rangle}
\newcommand{\op}[2]{\hat{\textbf{#1}}_{#2}}
\newcommand{\dagop}[2]{\hat{\textbf{#1}}_{#2}^\dag}
\usepackage[pdftex]{graphicx}
\usepackage{mathrsfs}
\usepackage[colorlinks]{hyperref}
\usepackage[dvipsnames]{xcolor}

\newcommand{\sihui}[1]{{\color{Orchid}{#1}}}
\newcommand{\peter}[1]{{\color{YellowGreen}{#1}}}
\newcommand{\comment}[1]{{\color{blue}{#1}}}
\newcommand{\ah}{\hat{a}}
\newcommand{\adagh}{\hat{a}^\dag}

\begin{document}

\bibliographystyle{apsrev}

%
% Title
%

\title{The Resurgence of the Linear Optics Interferometer --- Recent Advances \& Applications}

%
% Authors
%

\author{Si-Hui Tan}
\email[]{sihui\_tan@sutd.edu.sg}
\affiliation{Singapore University of Technology and Design, 8 Somapah Road, Singapore}

\author{Peter P. Rohde}
\email[]{dr.rohde@gmail.com}
\homepage{http://www.peterrohde.org}
\affiliation{Centre for Quantum Software \& Information (CQSI), Faculty of Engineering \& Information Technology, University of Technology Sydney, NSW 2007, Australia}

\date{\today}

\frenchspacing

%
% Abstract
%

\begin{abstract}
\end{abstract}

\maketitle

\tableofcontents

\section{Introduction}

\sihui{Si-Hui can colour code things she adds like this}

\peter{And Peter can do it like this}

\comment{Let's add comments and questions like this}

Technical advancements have been made on many fronts. It is possible now to put single-photon sources and linear-optical networks on a silica chip. The advantage of using such integrated photonics over bulk optics is that it is more stable against phase fluctuations, and miniaturized. This increases the scalability of optical implementations of quantum information protocols.


\section{Mathematical background}
\comment{Mathematical representation for LO networks, and very basic background on quantum optics}\\

A idealized single photon in a quantum interferometer is described by its creation operator $\adagh_{j}$, where $j$ is the label of the mode the photon is in  within the interferometer. The creation and annihilation operators satisfy the bosonic comutator relationship $[\ah_{j},\adagh_{k}]=\delta_{j,k}$. A similar commutator relationship can be  written up when more degrees of freedom, such as polarization, orbital angular momentum, and time-bins \cite{bib:Tillmann2015,bib:Bozinovic2013, bib:Nicolas2014, bib:Humphreys2013, bib:Donohue2013}, are present. When multiple photons are present, they experience quantum interference when all quantum labels are the same.

The action of a $2d$-port linear optical interferometer (with an equal number of input and output ports) is expressed as an application of unitary operations on the creation operators,

\begin{align}
b_i^{\dag}=\sum_{j=1}^d U_{ij}a_j^\dag \ ,
\end{align}
where $a_j^\dag$ and $b_i^\dag$ are the creation operators of a single input and output photon in the $j$-th and $i$-th modes respectively, and $U\in SU(d)$. All such transformation can be expressed as sequence of beamsplitters and waveplates \cite{bib:Reck1994}. In the case when photons have additional labels, for instance, if they have internal labels on top of spatial labels, it is also possible to derive an analogous decomposition, known as a cosine-sine decomposition \cite{bib:Dhand2015}, that realizes the unitary transformation on the photons into a sequence of beamsplitters and internal transformations. Toolkits using group theory are being developed to deal with partial distinguishabilities among interfering photons \cite{bib:Tan2013,bib:deGuise2014,bib:deGuise2015}. Others use
quantum-to-classical transitions to explain multiparticle interference \cite{bib:Ra2013}. \sihui{Experimental implementation [Walmsley, Jennewein]}



\section{Optical encoding of quantum information on single-photons}

%\subsection{Single-photons}

\subsubsection{Polarisation}

\subsubsection{Dual-rail}

\subsubsection{Time-bins}

%\subsection{Continuous-variables}

%\subsubsection{Coherent states}

%\subsubsection{Squeezed states}

\section{Efficient circuit decompositions of linear optics networks}

\comment{Discuss the Reck et al. decomposition}

The task of implementing an arbitrary quantum computation on linear optics comes down to implementing an arbitrary $n\times n$ unitary matrix. If a non-unitary transformation is desired, it can be embedded within a unitary matrix with larger dimensions. An algorithm for expressing an arbitrary unitary matrix {\it exactly} in terms of a sequence of $\mathcal{O}(n^2)$ beamsplitters and phase-shifters exists \cite{bib:Reck1994}. Alternatively, Mach-Zedner interferometer can also be used as building blocks instead of beamsplitters and phase shifters \cite{bib:Reck1994, bib:Englert2001}. Later, it has been shown that any nontrivial beam splitter, that does more than swapping modes around or add phases to them, is universal for linear optics \cite{bib:Bouland2014}. However, they do not provide a construction for arbitrary unitaries.

If the linear optical transformations can be realized on various degrees of freedom of light, then it is possible to realize a $n\times n$ arbitrary unitary transformation, where $n=n_s n_p$ for $n_s$ spatial modes, and $n_p$ internal modes, by a sequence of $\mathcal{O}(n_s^2 n_p)$ beamsplitters and $\mathcal{O}(n_s^2)$ internal transformations \cite{bib:Dhand2015}. Their approach reduces the required number of beamsplitters but increases the total number of optical elements needed increases by a factor of 2.

\section{Reconstructing the linear optical network}
\sihui{Discuss the role of quantum tomography here. 
Algorithms using O'Brien and Laing, Schaffner and Tillmann--using single-photon, and two-photon inputs to reconstruct the Euler angles of the circuit.
Others using coherent-state inputs which has a lower requirement on the state-preparation side.}

In many practical situations, the structure of a linear optical device in terms of its constituent beamsplitters and phase shifters is known once it is built. However, owing to manufacturing imperfections, a precise characterization of these devices is still needed post-production. One of the ways, this can be done is via a quantum process tomography \cite{bib:Mitchell03,bib:Obrien04,bib:Lobino08,bib:Saleh11} . However, quantum process tomography is an expensive method in terms of number of measurements required to characterize the network, and it becomes impractical for large optical networks which can now be as large as 900 modes (check citations for number of modes). Alternative characterization protocols have been developed using quantum interference of various quantum light sources \cite{bib:Laing12,bib:Rahimi-Keshari13} in the linear optical device.

Generally, the unitary matrix of the $d\times d$ linear optical device are complex numbers $U_{ij}=r_{ij}e^{i\theta_{ij}}$, where $0 \leq r_{ij}\leq 1$, and $0\leq \theta_{ij}\leq 2\pi$. The scheme in \cite{bib:Laing12} relies on injecting one-photon and two-photon states into the linear optical network with correlated photon detection. First, they note some equivalencies: two unitaries $U$ and $U'$ are equivalent if there exist two diagonal unitary matrices $D_1^U$ and $D_2^U$ such that $U'=D^U_1 U D^U_2$, because these diagonal matrices are regarded as unknown and trivial phases on the input and output ports of the network, to which the one-photon and two-photon data are insensitive to. This reduces the first row and first column elements to real numbers, i.e. $\theta_{1j}=\theta_{i1}=0$. Second, the photon statistics remain unchanged under the complex conjugation of $U$. Thus, the imaginary part of $M_{2,2}$ must be non-negative.

\section{Experimental implementation}

\subsection{State preparation}
\sihui{Cat state--Schoelkopf group}

\subsubsection{Single-photons}

\subsubsection{Bell pairs}

\subsubsection{Coherent states}

\subsubsection{Squeezed states}

\subsection{Linear optics networks}

\subsubsection{Bulk-optics}

\subsubsection{Waveguides}

\subsubsection{Time-bins}

\comment{Discuss fibre-loop architecture}

\subsection{Measurement}

\subsubsection{Photodetection}

\comment{Discuss number-resolved and bucket detectors, multiplexed detection, APDs, current micropillar detectors}

\subsubsection{Homodyning}

\section{Applications for linear optics interferometry}

\subsection{Linear optics quantum computation}

\subsection{Boson-sampling}

\subsection{Quantum metrology}

\comment{Discuss NOON states - Heisenberg limited}

\comment{Discuss MORDOR scheme}

\subsection{Encrypted quantum computation}

\section{State of the art}

\comment{Discuss where experiments are at at the moment}

\section{Conclusion}

%
% Acknowledgments
%

\begin{acknowledgments}
P.P.R. is funded by an ARC Future Fellowship (project FT160100397).
\end{acknowledgments}

%
% Bibliography
%

\bibliography{paper}

\end{document}
